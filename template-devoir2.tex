\documentclass[12pt,addpoints]{exam}
\usepackage[utf8]{inputenc}
\usepackage{lastpage}
\usepackage{amsmath}
\usepackage{amsfonts}
\usepackage{amssymb}
\usepackage{enumerate}
\usepackage{mdframed}
\usepackage{array}
\usepackage{graphics}
\usepackage{graphicx}
\usepackage{listings}
\usepackage{tikz}
\usetikzlibrary{arrows}
\usepackage{algpseudocode}
\usepackage{hyperref}

% Paramètres globaux
\bonuspointpoints{point \emph{bonus}}{points \emph{boni}}
\parindent 0cm
\hqword{Question}
\hpword{Sur}
\hsword{Note}
\htword{Total}

% Solutions
\printanswers

% En-tête et pieds de page
\headrule
\cfoot{\thepage /\pageref{LastPage}}
\lhead{INF5071 --- Infographie}
\rhead{Automne 2016}
\newcommand{\headrulewidth}{0.4pt}
\newcommand{\footrulewidth}{0.4pt}

% Algorithmes
\algrenewcommand\algorithmicwhile{\textbf{tant que}}
\algrenewcommand\algorithmicfunction{\textbf{fonction}}
\algrenewcommand\algorithmicend{\textbf{fin}}
\algrenewcommand\algorithmicdo{\textbf{faire}}
\algrenewcommand\algorithmicfor{\textbf{pour}}
\algrenewcommand\algorithmicif{\textbf{si}}
\algrenewcommand\algorithmicelse{\textbf{sinon}}
\algrenewcommand\algorithmicthen{\textbf{alors}}
\algrenewcommand\algorithmicreturn{\textbf{retourner}}
\algrenewcommand\algorithmicrepeat{\textbf{faire}}
\algrenewcommand\algorithmicuntil{\textbf{tant que}}
\newcommand{\enteteproc}[2]{\vskip\baselineskip \hspace*{\fill} \algorithmicprocedure~\Call{#1}{#2} \hspace*{\fill} \vskip\baselineskip}
\newcommand{\entetefunc}[3]{\vskip\baselineskip \hspace*{\fill} \algorithmicfunction~\Call{#1}{#2} : #3 \hspace*{\fill} \vskip\baselineskip}

% Code
\lstset{
  literate={é}{{\'e}}1
           {è}{{\`e}}1
           {ê}{{\^e}}1
           {î}{{\^i}}1
           {à}{{\`a}}1
           {É}{{\'E}}1,
  basicstyle=\footnotesize\ttfamily,
  keywordstyle=\bfseries\color{green!70!black},
  identifierstyle=\color{blue},
  commentstyle=\ttfamily\color{red!70!black}\emph,
  stringstyle=\bfseries\color{orange},
  showstringspaces=false,
  language=python,
  breaklines=true
}

% Raccourcis
\newcommand{\bigo}{\mathcal{O}}
\newcommand{\N}{\mathbb{N}}
\newcommand{\R}{\mathbb{R}}
\newcommand{\sign}{\mathrm{signe}}

\begin{document}
	
\ifprintanswers
  \vskip 1.2cm
  \centerline{\large\underline{\textbf{Solution du devoir 2}}}
  \vskip 0.5cm
  \begin{mdframed} \begin{center}
  \textbf{Auteur : Alexandre Blondin Massé} \\
  \end{center} \end{mdframed}
  \vskip 0.5cm
\else
  \vskip 1.2cm
  \centerline{\large\underline{\textbf{Devoir 2}}}
  \centerline{\small (à remettre au plus tard le 7 décembre 2016, à 16h00)}
  \centerline{\small (en classe ou dans la chute du département d'informatique, située au PK-4150)}
  \vskip 1.2cm
Le devoir doit être rédigé \textbf{individuellement} et à l'\textbf{ordinateur} (il est cependant permis de remettre des annexes avec des \textbf{schémas dessinés} manuellement). Vous devez \textbf{justifier} chacune de vos réponses. Tout retard entraînera une pénalité de \textbf{20\%} par jour ouvrable.
\fi

\ifprintanswers
\else
\begin{center} \gradetable[h] \end{center}
\fi

\begin{questions}

% Tore
\question
Un tore $T(R,r)$ est une surface qui peut être paramétrée par la fonction
$$\vec{s}(u,v) = ((R + r\cos v) \cos u, (R + r\cos v) \sin u, r\sin v),$$
où $r$ est le rayon du petit cercle qui engendre le tore et $R$ est la distance entre le centre du tube et le centre du tore. Donnez le pseudocode (ou écrivez un programme) qui prend un point quelconque $(x,y,z)$ de l'espace et qui retourne une des deux valeurs suivantes~:
\begin{itemize}
  \item Le vecteur nul si $(x,y,z)$ ne se trouve pas sur le tore $T(R,r)$;
  \item Le vecteur normal à la surface au point $(x,y,z)$ si $(x,y,z)$ se trouve sur le tore $T(R,r)$.
\end{itemize}
N'hésitez pas à consulter des références pour vous aider à répondre à cette question, mais n'oubliez pas de citer vos sources.
\begin{solution}
Ajouter votre solution ici
\end{solution}

% Lancer de rayon
\question
Considérez une scène contenant les éléments suivants~:
\begin{itemize}
  \item Un cube centré en $(0,0,0)$ dont les côtés mesurent $10$ unités;
  \item Deux sphères de rayon $2$ centrées respectivement en $(3,0,0)$ et $(-3,0,0)$.
\end{itemize}
Un rayon est lancé à partir du point $(0,4,4)$ en direction du point $(3,0,0)$. Décrivez la trajectoire de ce rayon s'il est parfaitement réfléchi chaque fois qu'il entre en collision avec un objet de la scène (incluant les parois du cube), en supposant qu'il disparaît lorsqu'il entre en collision une cinquième fois avec un objet. Vous pouvez répondre à cette question de deux façons~:
\begin{itemize}
  \item En présentant les calculs mathématiques;
  \item En présentant un programme dans le langage de votre choix qui effectue le calcul.
\end{itemize}
\begin{solution}
Ajouter votre solution ici
\end{solution}

% Détection de collisions
\question
Une boule $B$ de centre $(x,y,z)$ et de rayon $r$ évolue dans une scène fermée $S$ de la forme $(h \pm a,k \pm b,\ell \pm c)$ qui contient $n$ plateformes $P_i$ de la forme $(h_i \pm a_i,k_i \pm b_i,\ell_i \pm c_i)$, pour $i = 1,2,\ldots,n$, où $(h_i,k_i,\ell_i)$ est le centre de la plateforme $P_i$, et $(a_i,b_i,c_i)$ est un triplet décrivant la longueur d'un demi-côté de la plateforme $P_i$.

Donnez le pseudocode d'une fonction
\entetefunc{Collision}{$B$ : boule, $S$ : scène, $P$ : liste de plateformes} : booléen
qui retourne vrai si et seulement si la boule $B$ est en collision avec une des parois de la scène $S$ ou une des plateformes contenues dans $P$. N'oubliez pas de citer vos sources. \emph{Aide~:} Théorème des axes séparants.
\begin{solution}
Ajouter votre solution ici
\end{solution}

% Texture 3D
\question
L'objectif de cette question est de concevoir une texture 3D et de l'appliquer à un modèle. Vous devez utiliser le moteur de rendu \emph{Cycles} du logiciel Blender.
\begin{parts}
  \part[20] Dans un premier temps, concevez un modèle en deux versions~:
  \begin{itemize}
    \item Une première version avec un niveau de détails élevé, en utilisant le mode ``Sculpt'' de Blender;
    \item Une seconde version avec un niveau de détails plus bas. Ce deuxième modèle devrait être créé après le premier. Il existe plusieurs façons de créer une version avec moins de polygones (voir par exemple \href{http://blender.stackexchange.com/questions/6253/how-to-convert-from-high-poly-to-low-poly}{\color{orange}cette discussion}).
  \end{itemize}
  Ensuite, à l'aide de Blender, générez l'image des normales (\textit{normal map}) avec l'opération ``Bake'' du modèle le plus détaillé. Présentez vos modèles ainsi que l'image des normales.
  \part[20] Ensuite, créez une texture de façon procédurale (par exemple, du marbre, du bois, de l'écorce d'arbre, du métal, etc.) à l'aide du ``Node Editor'' de Blender et présentez le résultat.
  \part[20] Toujours avec l'éditeur de noeuds, appliquez votre texture au modèle le moins détaillé en prenant en compte l'image des normales. Présentez le graphe de noeuds vous permettant de générer votre texture. N'hésitez pas à ajouter (c'est optionnel) des noeuds qui tiennent compte des reflets de lumière (\textit{glossy} ou encore \textit{ambiant occlusion map}).
\end{parts}
N'oubliez pas de citer vos sources.
\begin{solution}
Ajouter votre solution ici
\end{solution}

\end{questions}

\end{document}
